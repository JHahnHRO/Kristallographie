% !TeX spellcheck = de_DE
% !TeX root = kristallographie_skript.tex
\begin{definition}
Eine \udot{Gruppe} $(G,\circ)$ besteht aus
\begin{itemize}
\item einer Menge $G$ sowie
\item einer Abbildung $\circ: G\times G\to G$, d.h. einer Verknüpfung, die aus zwei Gruppenelementen $g_1,g_2\in G$ ein neues Gruppenelement $g_1\circ g_2$ macht
\end{itemize}
die die folgenden Eigenschaften erfüllen:
\begin{enumerate}[label=(G\arabic*)]
\item Assoziativität, d.h. wir dürfen in zusammengesetzten Ausdrücken beliebig umklammern:
\[\forall x,y,z\in G: x\circ(y\circ z) = (x\circ y)\circ z\]
In der Praxis bedeutet das, dass wir Klammern einfach weglassen und z.B. $x\circ y\circ z$ schreiben.
\item Neutrales Element: Es gibt (genau) ein Element, das gar nichts tut, wenn wir es mit anderen Gruppenelementen verknüpfen:
\[\exists 1_G\in G \forall x\in G: 1_G\circ x = x = x\circ 1_G\]
\item Inverse Elemente: Jede durch ein Gruppenelement repräsentierte Aktion kann durch (genau) ein anderes Gruppenelement rückgängig gemacht werden:
\[\forall x\in G\exists x^{-1}\in G: x\circ x^{-1}=1_G=x^{-1}\circ x\]
\end{enumerate}
Manche Gruppen erfüllen zusätzliche Eigenschaften, z.B. wird eine Gruppe \udot{kommutativ} oder \udot{abelsch} genannt, wenn sie
\begin{enumerate}[label=(G\arabic*),resume]
\item Kommutativität: Es ist egal, in welcher Reihenfolge wir Elemente verknüpfen:
\[\forall x,y\in G: x\circ y = y\circ x\]
\end{enumerate}
erfüllt.

Wenn aus dem Kontext klar ist, welche Verknüpfung $\circ$ sein soll oder wenn die Verknüpfung einer generischen Gruppe gemeint ist, schreibt man sie meistens als Multiplikation, d.h. man schreibt $g\cdot h$ oder gar $gh$ anstelle von $g\circ h$.
\end{definition}

\begin{example}[Symmetriegruppen]
Wir beschäftigen uns mit Gruppen, weil sie Symmetrien von Objekten beschreiben. Für jedes geometrische oder abstrakt-mathematische Objekt $X$ gibt es eine Gruppe $\Aut(X)$, die alle im jeweiligen Kontext relevanten Transformationen umfasst, welche $X$ nicht verändern. Die Gruppenverknüpfung $\circ$ ist in diesen Beispielen die Hintereinanderausführung von Transformation, d.h. $f\circ g$ ist die Operation \enquote{$f$ nach $g$}, also diejenige Transformation, die man erhält, wenn man zuerst $g$ und dann $f$ anwendet. Das neutrale Element in diesen Beispielen ist immer die identische Transformation $\id$, also diejenige, die alles so lässt wie es ist.

\begin{enumerate}
\item $\Isom(\IR^n)$ ist eine Gruppe, denn die Verknüpfung von zwei Isometrien ist wieder eine Isometrie, $\id$ ist eine Isometrie, jede Isometrie ist bijektiv und die Umkehrabbildung einer Isometrie ist selbst eine Isometrie.
\item Ist $X$ etwa eine Punktmenge im $\IR^n$, z.B. ein Polyeder oder ein Kristall, so ist $\Aut(X)$ die Gruppe aller starren (=abstandserhaltenden) Bewegungen, die $X$ unverändert lassen:
\[\Aut(X) := \Set{g\in\Isom(\IR^n) | g(X) = X}\]
\item Ist z.B. $X$ ein gleichseitiges Dreieck in der Ebene, dann hat $\Aut(X)$ genau sechs Elemente:

Die Identität, d.h. die Drehung um $\SI{0}{\degree}$, die Drehung um $\SI{120}{\degree}$, die Drehung um $\SI{240}{\degree}$ sowie drei Spiegelungen an den drei möglichen Spiegelachsen jeweils durch einen Eckpunkt und den gegenüberliegende Seitenmittelpunkt.
\item Ist $X$ einfach irgendeine Menge, dann bezeichnet man mit $\Sym(X)$ die \udot{symmetrische Gruppe auf/von $X$}. Da eine beliebige Menge völlig unstrukturiert ist, werden einfach \emph{alle} Abbildungen betrachtet, d.h.
\[\Sym(X) := \Set{f: X\to X | f\text{ ist bijektiv}}\]
(Bijektivität ist wichtig, damit wir wirklich inverse Element bekommen. Beliebige Abbildungen sind nicht invertierbar. Im geometrischen Beispiel brauchten wir das nicht, da starre Bewegungen immer invertierbar sind: Jede Verschiebung, Drehung, Punkt- oder Ebenenspiegelung kann durch eine Verschiebung, Drehung, Punkt- bzw. Ebenenspiegelung rückgängig gemacht werden)

Speziell, wenn $X$ eine endliche Menge ist, dann bezeichnet man bijektive Abbildungen $X\to X$ auch als \udot{Permutationen} und die Gruppe $\Sym(X)$ auch als \udot{Permutationsgruppe}. Es gibt genau $\abs{\Sym(X)} = \abs{X}!$ viele Permutationen.

\item Ist z.B. $X=\set{1,2,3}$, dann gibt es genau $3!=6$ Permutationen dieser Menge:

Die Identität
\[1\mapsto 1, 2\mapsto 2, 3\mapsto 3\]
drei \udot{Transpositionen}, d.h. Permutationen, die genau zwei Elemente tauschen:
\[1\mapsto 2, 2\mapsto 1, 3\mapsto 3\]
\[1\mapsto 3, 2\mapsto 2, 3\mapsto 1\]
\[1\mapsto 1, 2\mapsto 3, 3\mapsto 2\]
sowie zwei \udot{3-Zyklen}, die Permutationen, die drei Elemente im Kreis permutieren:
\[1\mapsto 2, 2\mapsto 3, 3\mapsto 1\]
\[1\mapsto 3, 2\mapsto 1, 3\mapsto 2\]
\end{enumerate}
\end{example}

\begin{example}[Abstrakte Gruppen]
Es gibt Gruppen, denen man nicht sofort ansieht, dass sie Symmetrien beschreiben. Es gibt auch (wenige) Gruppen, die gar keine Symmetrien beschreiben.

\begin{enumerate}
\item Zahlenbereiche mit Addition: $\IZ,\IQ,\IR,\IC,\IR^n$ jeweils zusammen mit $\circ=+$ sind kommutative Gruppen. Das neutrale Element ist die Zahl Null bzw. der Nullvektor. Inverse Elemente sind Negative.
\item Zahlenbereiche mit Multiplikation: $\IQ\setminus\set{0},\IR\setminus\set{0},\IC\setminus\set{0}$ sind jeweils zusammen mit $\circ=\cdot$ kommutative Gruppen. Das neutrale Element ist die Zahl Eins. Inverse Elemente sind Reziproke.
\end{enumerate}

Gruppen dienen also gleichzeitig der gemeinsamen Beschreibung (einiger) der Eigenschaften, die die uns bekannten Grundrechenarten erfüllen. Sie werden ebenso benutzt, um die Eigenschaften von anderen Strukturen zu beschreiben, die sich in bestimmten Aspekten ähnlich verhalten wie die uns bekannten Zahlenbereiche sich bzgl. Addition und Multiplikation verhalten (sogenannte Ringe und Körper).
\end{example}

\begin{definition}[Untergruppen]
Kennen wir bereits eine Gruppe $(G,\circ)$ und ist $U\subseteq G$ eine Teilmenge mit den folgenden Eigenschaften:
\begin{enumerate}[label=(UG\arabic*)]
\item $U$ enthält das neutrale Element:
\[1_G\in U\]
\item $U$ ist unter Multiplikation abgeschlossen:
\[\forall x,y\in U: x\circ y\in U\]
\item $U$ ist unter Inversenbildung abgeschlossen:
\[\forall x\in U: x^{-1}\in U\]
\end{enumerate}
Dann ist $U$ selbst eine Gruppe mit der gleichen Verknüpfung. Wir schreiben für diesen Sachverhalt $U\leq G$, um deutlich zu machen, dass es sich nicht um eine beliebige Teilmenge handelt, sondern um eine Untergruppe.
\end{definition}

\begin{example}
Dies tritt häufig auf, wenn wir nicht alle Symmetrien eines Objekts betrachten, sondern nur Symmetrien eines bestimmten Typs.

\begin{enumerate}
\item Die orientierungserhaltenden Bewegungen (d.h. solche, die Rechtssysteme wieder auf Rechtssysteme abbilden) $\IR^n\to\IR^n$ bilden eine Untergruppe von $\Isom(\IR^n)$. Eine Spiegelung ist nicht orientierungserhaltend, eine Drehung oder Translation schon.
\item Die Translationen bilden eine Untergruppe von $\Isom(\IR^n)$.
\item Haben wir einen Nullpunkt gewählt, dann sind die orthogonalen Abbildungen eine Untergruppe von $\Isom(\IR^n)$.
\end{enumerate}
\end{example}

\begin{theoremdef}[Satz von Lagrange]
Sei $G$ eine Gruppe und $U\leq G$ eine Untergruppe. Eine Teilmenge der Form
\[gU := \Set{gu | u\in U}\]
für ein $g\in G$ heißt \udot{(Links)Nebenklasse von $U$ in $G$}. Die Anzahl aller Linksnebenklassen wird mit $\abs{G:U}$ bezeichnet und heißt \udot{Index von $U$ in $G$}.

Die Teilmengen haben die folgenden Eigenschaften:
\begin{enumerate}
\item Alle Nebenklassen von $U$ sind gleich groß: $\forall g\in G: \abs{gU} = \abs{U}$
\item Zwei Nebenklassen sind entweder identisch oder disjunkt.
\item Ist $G$ endlich, so gilt $\abs{G:U} = \frac{\abs{G}}{\abs{U}}$.
\end{enumerate}
\end{theoremdef}

\begin{example}
\begin{enumerate}
\item Wir fixieren einen Nullpunkt und betrachten $G=\Isom(\IR^n)$ und $U=O(\IR^n)$. Wenn wir einen Punkt $y\in\IR^n$ haben, dann ist
\[\Set{h\in G | h(0)=y}\]
eine Nebenklasse von $U$. Es gibt immer mindestens eine Bewegung, die $0$ auf $y$ abbildet, nämlich die Translation $\tau_y$. Ist $g$ irgendeine Bewegung mit $g(0)=y$, dann gilt
\[h(x)=y \iff g^{-1}(h(0)) = g^{-1}(y) = 0 \iff g^{-1}h\in U \iff h=g(g^{-1}h)\in gU\]
also ist $\Set{h\in G | h(x)=y} = gU$.
\item Sei $G=\Isom(\IR^n)$ und $U=\Isom_0(\IR^n)$ die Untergruppe aller orientierungserhaltenden Bewegungen. Dann ist die Menge aller orientierungs\emph{umkehrenden} Bewegungen eine Nebenklasse von $U$.
\end{enumerate}
\end{example}

\begin{proof}
a. $U\to gU, u\mapsto gu$ ist eine Bijektion, denn $gU\to U, x\mapsto g^{-1} x$ ist eine inverse Abbildung. Wenn es eine Bijektion zwischen zwei Mengen gibt, dann sind sie gleich groß.

\medbreak
b. Betrachte $g,h\in G$ und die beiden Nebenklassen $gU$ und $hU$. Wenn sie disjunkt sind, dann sind wir schon fertig. Wenn $gU$ und $hU$ nicht disjunkt sind, d.h. es gibt ein $x\in gU\cap hU$. Dann muss es laut Definition zwei Elemente $u_1,u_2\in U$ geben, sodass $x=gu_1$ sowie $x=hu_2$ gilt. Wenn man nun von rechts mit einem beliebigen Element $u\in U$ multipliziert, findet man, dass $xu = g(u_1 u)\in gU$ und $xu=h(u_2 u)\in hU$ ist. Also folgt $xU\subseteq gU$ und $xU\subseteq hU$.

Umgekehrt gilt aber auch $g=xu_1^{-1}$ und $h=xu_2^{-1}$. Mit der gleichen Überlegung folgt also auch $gU\subseteq xU$ und $hU\subseteq xU$. Setzen wir beide Erkenntnisse zusammen, so finden wir $gU=xU=hU$.

\medbreak
c. Wenn $G$ endlich ist, dann ist $\abs{U}$ sowie die Anzahl der Nebenklassen $\abs{G:U}$ auch endlich. Jedes Element von $g$ ist in mindestens einer Nebenklasse enthalten, nämlich in $gU$ (denn $g=g1$ und $1\in U$). Andererseits kann es nicht in mehr als einer Nebenklasse enthalten sein, denn die sind ja alle disjunkt, wie wir soeben herausgefunden haben. Also ist jedes Element von $G$ in \emph{genau einer} Nebenklasse enthalten, d.h. wenn $g_1 U$, $g_2 U$, ..., $g_k U$ eine vollständige Auflistung aller Nebenklassen ist (d.h. $k=\abs{G:U})$, dann muss
\[\abs{G} = \abs{g_1 U} + \abs{g_2 U} + \cdots + \abs{g_k U}\]
gelten. In a. haben wir jedoch gesehen, dass alle diese Summanden die gleiche Zahl sind, nämlich $\abs{U}$. Also folgt $\abs{G}=\abs{U}+\abs{U}+\cdots+\abs{U} = k\abs{U} = \abs{G:U}\abs{U}$.
\end{proof}

\begin{definition}
Ist $G$ eine endliche Gruppe, so nennt man ihre Größe $\abs{G}$ auch \udot{Ordnung der Gruppe}. Ist $g\in G$ ein Element von $G$, so nennt man die Ordnung der Untergruppe $\langle g\rangle:=\set{g^k | k\in\IZ}$ auch \udot{Ordnung des Elements}.
\end{definition}

\begin{example}
\begin{enumerate}
\item Die Identität hat Ordnung 1.
\item Spiegelungen sind Elemente der Ordnung 2.
\item Eine Drehung um den Winkel $\frac{k}{n}2\pi$, wobei $\frac{k}{n}$ vollständig gekürzt ist, hat genau Ordnung $n$.
\item Eine Drehinversion um einen Winkel $\frac{k}{n}2\pi$, wobei $\frac{k}{n}$ vollständig gekürzt ist, hat Ordnung $n$, wenn $n$ gerade ist, und Ordnung $2n$, wenn $n$ ungerade ist.
\item Translationen, Schraubungen im Raum und Gleitspiegelungen in der Ebene, die einen Verschiebungsanteil $\neq 0$ haben, haben jeweils unendliche Ordnung.
\end{enumerate}
\end{example}

\begin{remark}
Aufgrund des Satzes von Lagrange sind Elementordnungen immer Teiler der Gruppenordnung. Wenn wir also in einer Symmetriegruppe bestimmte Symmetrien sofort sehen, z.B. eine Drehung um $\SI{72}{\degree}=\frac{2\pi}{5}$ und eine um $\SI{120}{\degree}=\frac{2\pi}{3}$, dann wissen wir, dass die gesamte Symmetriegruppe eine durch $\operatorname{kgV}(3,5)=15$ teilbare Ordnung (oder $\infty$) haben muss.
\end{remark}

