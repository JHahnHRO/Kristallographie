% !TeX spellcheck = de_DE
% !TeX root = kristallographie_skript.tex

Jetzt haben wir alles Wissen beisammen, um einen Kristall, also die Lage aller Atome im Kristallgitter beschreiben zu können. Dazu definieren wir eine Basis und geben die Art und Position jedes Atoms in der Basiszelle mit Hilfe der Basis an. Zur Definition der Basis werden meist die drei Längen und Winkel angegeben, manchmal werden die Basisvektoren auch im Verhältnis zu irgendeiner vorher definierten Standardbasis angegeben. Die Art und Position von jedem Atom ist meist als Liste mit vier Einträgen anzutreffen: Kernladung (Ordnungszahl), $x$,$y$,$z$, wobei $x,y,z \in [0,1)$ die Koordinaten des Atoms sind. Diese Liste ist das Motiv unseres Gitters, kann ganz schön lang werden und symmetrieäquivalente Atome beinhalten. Daher wird oft auch das Hermann-Mauguin-Symbol zusammen mit der Basis und einer minimalen Liste angegeben. Mit Hilfe der Symmetrieoperationen kann dann aus der minimalen Liste die vollständige Basiszelle rekonstruiert werden.