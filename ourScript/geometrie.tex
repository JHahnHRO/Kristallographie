% !TeX spellcheck = de_DE
% !TeX root = kristallographie_skript.tex
\begin{definition}
Eine \udot{starre Bewegung} oder \udot{Isometrie} ist eine Abbildung $\phi:\IR^n\to\IR^n$, die abstandserhaltend ist, d.h.
\[\forall x,y\in\IR^n: \norm{\phi(x)-\phi(y)} = \norm{x-y}\]
\end{definition}

\begin{example}
\begin{enumerate}
\item Translationen: $\tau_v(x) := x+v$.
\item Drehungen: Bis auf Koordinatenwahl die Abbildungen der Form
\[\begin{pmatrix}x\\y\\z\end{pmatrix}\mapsto\begin{pmatrix}
\cos(\alpha)x + (-\sin(\alpha)) y \\
\sin(\alpha)x + \cos(\alpha)y \\
z
\end{pmatrix}\]
$\alpha$ wird dabei als Drehwinkel bezeichnet. Die Drehachse erkennt man daran, dass es eine Gerade $A\subseteq\IR^3$ ist, die $D(A)=A$ erfüllt. In obiger Beschreibung ist das Koordinatensystem so gewählt worden, dass die Drehachse genau die Gerade $\Set{\begin{psmallmatrix}0\\0\\z\end{psmallmatrix} | z\in\IR}$ ist.
\item Inversionen = Punktspiegelungen
\item Spiegelungen an einer Geraden (in $\IR^2$) bzw. an einer Ebene (in $\IR^3$) bzw. allgemein an einer Hyperebene ($(n-1)$-dimensionale Unterräume von $\IR^n$).
\item Wenn $f,g$ Isometrien sind, dann auch $f\circ g$, z.B.
\begin{enumerate}
\item Gleitspiegelungen: Im $\IR^2$ eine Spiegelung an einer Geraden gefolgt von einer Translation in Richtung derselben Geraden.
\item Drehinversionen: Eine Drehung $g$ gefolgt von einer Inversion $f$ in einem Punkt auf der Drehachse von $g$.
\item Schraubungen: Eine Drehung $g$ gefolgt von einer Translation $f$ entlang der Drehachse.
\end{enumerate}
\item Die Identität $\id: x\mapsto x$. Das ist gleichzeitig die Translation um die Distanz $0$ (in jede Richtung) und die Drehung um den Winkel $0$ (mit jeder Drehachse).
\end{enumerate}
\end{example}

\begin{definition}
Angenommen, wir haben einen Nullpunkt gewählt. Eine Isometrie, die den Nullpunkt fixiert, wird \udot{orthogonale} Abbildung genannt.
\end{definition}

\begin{theorem}
Es sei $f:\IR^n\to\IR^n$ eine Bewegung. Dann gilt:
\begin{enumerate}
\item $f$ ist \udot{affin}, d.h.
\[\forall p_0,p_1\in\IR^n, \lambda\in\IR: f(\lambda p_1 + (1-\lambda)p_0) = \lambda f(p_1)+(1-\lambda)f(p_0)\]
\item Haben wir einen Nullpunkt gewählt und ist $f$ orthogonal, dann ist $f$ sogar \udot{linear}, d.h.
\begin{enumerate}
\item $\forall v_1,v_2\in\IR^n: f(v_1+v_2)=f(v_1)+f(v_2)$
\item $\forall v\in\IR^n, \lambda\in\IR: f(\lambda v)=\lambda f(v)$.
\end{enumerate}
\item Haben wir einen Nullpunkt gewählt, dann ist ein beliebiges $f$ eine orthogonale Abbildung gefolgt von einer Translation.
\item $f$ erhält das Skalarprodukt von Vektoren und Winkel zwischen Vektoren.
\end{enumerate}
\end{theorem}
\begin{proof}
a. Sei zunächst $0\leq \lambda\leq 1$. Der Punkt $\lambda p_1+(1-\lambda)p_0 = p_0+\lambda(p_1-p_0) = p_1-(1-\lambda)(p_1-p_0)$ ist genau derjenige Punkt, der genau Abstand $\lambda\norm{p_1-p_0}$ von $p_0$ und $(1-\lambda)\norm{p_1-p_0}$ von $p_1$ hat und es gibt nur einen solchen Punkt. Daher ist $f(\lambda p_1+(1-\lambda)p_0)$ genau derjenige Punkt, der den Abstand $\lambda\norm{p_1-p_0}=\lambda\norm{f(p_1)-f(p_0)}$ von $f(p_0$ und den Abstand $(1-\lambda)\norm{p_1-p_0} = (1-\lambda)\norm{f(p_1)-f(p_0)}$ hat. Weil es nur einen solchen Punkt gibt, muss $f(\lambda p_1+(1-\lambda)p_0) = \lambda f(p_0)+(1-\lambda)f(p_1)$ sein.

Im allgemeinen Fall liegen die drei Punkte $p_0$, $p_1$ und $p_\lambda:=\lambda p_1+(1-\lambda)p_0$ ja auf einer gemeinsamen Geraden. Es gibt also immer einen, der zwischen den anderen zweien positioniert ist. Wenn das z.B. $p_0$ ist, dann gibt es eine Gleichung der Form $p_0 = \alpha p_\lambda + (1-\alpha)p_1$ mit $\alpha\in [0,1]$ (Übung: Man berechne $\alpha$ aus $\lambda$ und umgekehrt). Wenn $p_1$ zwischen $p_0$ und $p_\lambda$ liegt, gibt es analog eine Gleichung der Form $p_1=\beta p_\lambda + (1-\beta)p_1$ mit $\beta\in[0,1]$. In beiden Fällen folgt aus dem schon bewiesenen auch die gleiche Gleichung für $f(p_0)$, $f(p_1)$ und $f(p_\lambda)$.

\medbreak
b. Wenn $f(0)=0$ ist, dann können wir $p_0=0$ und $p_1=v$ in a. einsetzen und erhalten direkt, dass $f(\lambda v)=\lambda f(v)$ gilt.

Daraus folgern wir nun:
\[f(v_1+v_2) = 2f(\tfrac{1}{2}(v_1+v_2)) = 2f(\tfrac{1}{2}v_1 + \tfrac{1}{2}v_2) \overset{a.}{=} 2(\tfrac{1}{2}f(v_1)+\tfrac{1}{2}f(v_2))=f(v_1)+f(v_2)\]

\medbreak
c. Ist $f(0)=:v$, dann betrachten wir die Translation $\tau_v$. Dann ist nämlich $(\tau_v^{-1}\circ f)(0) = 0$, also ist $f=\tau_v\circ(\tau_v^{-1}\circ f)$ eine Komposition aus der orthogonalen Abbildung $\tau_v^{-1}\circ f$ gefolgt von der Translation $\tau_v$.

\medbreak
d. folgt aus $\tfrac{1}{2}(\norm{v_1+v_2}^2-\norm{v_1}^2-\norm{v_2}^2)=\braket{v_1,v_2}$, $\norm{v}=\norm{f(v)}$ und b.

Winkelerhaltung folgt dann aus $\cos(\sphericalangle(v_1,v_2)) = \frac{\braket{v_1,v_2}}{\norm{v_1}\cdot\norm{v_2}}$.
\end{proof}

\begin{remark}
Aus der Linearität und der Tatsache, dass wir nur Räume endlicher Dimension betrachten, kann man folgern, dass jede Isometrie bijektiv ist, also eine Umkehrabbildung besitzt. Die Umkehrabbildung muss dann selbst wieder eine Isometrie sein, denn:
\[\norm{p_1-p_2} = \norm{f(f^{-1}(p_1))-f(f^{-1}(p_2))} = \norm{f^{-1}(p_1)-f^{-1}(p_2)}\]
\end{remark}

\begin{theorem}[Klassifikation von Bewegungen in kleinen Dimensionen]
Alle Bewegungen des
\begin{enumerate}
\item $\IR^1$ sind Translationen oder Spiegelungen.
\item $\IR^2$ sind Drehungen oder Geradenspiegelungen gefolgt von einer Translation (ggf. um den Abstand $0$).
\item $\IR^3$ sind Drehungen, Ebenenspiegelungen oder Drehinversionen (ggf. um den Drehwinkel $0$) gefolgt von einer Translation (ggf. um den Abstand $0$).
\end{enumerate}
\end{theorem}