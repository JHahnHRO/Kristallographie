% !TeX spellcheck = de_DE
% !TeX root = kristallographie_skript.tex


Die Morphologie (das makroskopische Aussehen) der Kristalle wird von der mikroskopischen Anordnung der Atome bestimmt. Makroskopisch erkennbare Symmetrien treten im Kristallgitter und somit auch in der Basiszelle auf. Wir beschreiben die Morphologie eines (in Annahme schön und daher als konvexer Körper gewachsenen) Kristalls durch seine
\begin{itemize}
	\item ebene Begrenzungsflächen, eine Menge symmetrieäquivalenten Flächen bezeichnen wir als \udot{Flächenform}. 
	\item Kanten, eine Menge aller symmetrieäquivalenten Kanten bezeichnen wir als \udot{Schar von Gittergeraden/Gitterrichtungen}
\end{itemize}

\begin{definition}
	Die Menger aller an einem Kristall auftretenden Flächenformen bezeichnen wir als \udot{Tracht} des Kristalls.
\end{definition}
\begin{definition}
	Der \udot{Habitus} eines Kristalls beschreibt das Erscheinungsbild eines Kristalls, welches durch das  relative Größenverhältnis der auftretenden Flächenformen bestimmt wird.
	Kristalle mit gleicher Tracht können unterschiedlichen Habitus aufweisen. 
\end{definition}

Das liegt daran, wie unsere schönen Kristalle wachsen: Das Wachstum fängt an einem winzigen Kristallisationskeim (eine Unreinheit, irgendeine andere Art von Oberfläche) an, es bildet sich ein kleiner Kristallkeim und dieser expandiert in seine Umgebung hinein. Dabei lagern sich Atome aus der Umgebung an den Keim an. Wenn man nicht zu genau hinschaut wächst der Kristallkeim in alle Richtungen gleich, also wie eine größer werdende Kugel. Dies ist aber nicht die Realität: Je nachdem, um welche Atome und um welche Anordnung es sich handelt, lagern sich die Atome von der Umgebung schneller an manchen Stellen an, als an andere. Der Keim wächst also in manchen Richtungen schneller, als in andere. Wenn der Kristallisationskeim in allen Richtungen mit dem Material umgeben ist, aus dem der wachsende Kristall die Atome bezieht (dies kann eine Lösung, z.B. Honig, oder auch Schmelze sein), dann wachsen alle symmetrieäquivalente Flächen gleich schnell. Manche Flächenformen wachsen aber schneller als andere, daher können verschiedene Kristalle mit gleichem Hermann-Mauguin-Symbol immernoch unterschiedlichen Habitus aufweisen (z.B. Würfel, Oktaeder und Kubooktaeder).

