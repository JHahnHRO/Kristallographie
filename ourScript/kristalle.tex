
\begin{definition}[Kristalle]
Ein \udot{Kristall} (auch \udot{Kristallgitter}) ist eine Punktmenge $\Lambda\subseteq\IR^3$ (gedacht als die Menge aller Atome im Kristall) ...
\begin{enumerate}
\item ... die Translationssymmetrie hat, d.h. es gibt Vektoren $v_1,v_2,v_3\in\IR^3$ in drei unabhängige Richtungen, sodass immer, wenn $a\in\Lambda$ ein Atom im Kristall ist, $a+k_1v_1+k_2v_2+k_3v_3$ auch ein Atom im Kristall ist für alle ganzen Zahlen $k_1,k_2,k_3\in\IZ$.
\item ... die aus isolierten Punkten besteht, d.h. es gibt einen Mindestabstand $\delta>0$, sodass sich keine zwei Punkte $x,y\in\Lambda$ näher als $\delta$ kommen: $x\neq y \implies\norm{x-y}\geq\delta$.
\end{enumerate}
Streng genommen müssten wir verschiedene Sorten von Atome unterscheiden, die im Kristall vorkommen, z.B. nach ihrem Element, d.h. zu einem Kristallgitter könnte auch eine Funktion gehören, die jedem Punkt $a\in\Lambda$ ein Unterscheidungsmerkmal zuordnet, z.B. eine Zahl (man denke: Nr. im Periodensystem) zuordnet und
\begin{enumerate}[resume]
\item Die Punkte $a+k_1v_1+k_2v_2+k_3v_3$ für $k_1,k_2,k_3\in\IZ$ haben alle dieselbe Nummer.
\end{enumerate}
erfüllt. Es ist aber für Symmetriebetrachtungen ausreichend, nur ein-atomige Kristalle zu betrachten. Erst, wenn es um die Chemie dahinter geht, werden die tatsächlich auftretenden Elemente im Kristall wichtig.
\end{definition}

\begin{remark}
Insbesondere bedeutet die Bedingung des Mindestabstands, dass es nur abzählbar viele Punkte im Gitter gibt.
\end{remark}

\begin{definition}
Die \udot{Symmetriegruppe} oder auch \udot{Raumgruppe} eines Kristalls $\Lambda$ ist die Gruppe aller starren (=abstandserhaltenden) Bewegungen, die das Gitter in sich selbst abbilden:
\[\Aut(\Lambda) := \Set{s\in \operatorname{Isom}(\IR^3) | s(\Lambda)=\Lambda}\]
Nach Definition enthält $\Aut(\Lambda)$ mindestens die drei Translationen $x\mapsto x+t_i$. Die Menge \emph{aller} Translationen, die $\Lambda$ in sich selbst abbilden, sind eine Untergruppe von $\Aut(\Lambda)$.
\end{definition}

\begin{definition}
Ist $\Lambda$ ein Kristall, dann ist
\[Trans(\Lambda) := \Set{v\in\IR^3 | \forall a\in\Lambda: a+v\in\Lambda}\]
das \udot{Translationsgitter} des Kristalls.
\end{definition}

\begin{remark}
Weil die Punkte in $\Lambda$ einen Mindestabstand haben, haben die Elemente des Translationsgitters denselben (vielleicht sogar einen größeren) Mindestabstand. Man kann daraus folgern (wir werden es aber nicht tun), dass $Trans(\Lambda)$ unabhängige Vektoren $v_1,v_2,v_3$ enthält, sodass
\[Trans(\Lambda) = \IZ v_1+\IZ v_2+\IZ v_3\]
gilt. Dies können, müssen aber nicht, die drei Vektoren aus der Definition sein.
\end{remark}

\begin{definition}
Es sei $\Lambda$ ein Kristallgitter und $T=\Set{\tau_v | v\in Trans(\Lambda)}$ die Gruppe aller Translationen, die $\Lambda$ invariant lassen.

Eine \udot{Basiszelle von $\Lambda$} ist ein Paar $(Z,A)$ bestehend aus einem (konvexer, kompakter) Polyeder $Z\subseteq\IR^3$ und einer Punktmenge $M\subseteq Z$, sodass
\begin{enumerate}
\item ... die Translate von $M$ ganz $\Lambda$ überdecken, d.h. $\Lambda=\bigcup_{\tau\in T} \tau(M)$.
\item ... die Translate von $Z$ ganz $\IR^3$ überdecken, d.h. $\IR^3 = \bigcup_{\tau\in T} \tau(Z)$.
\item ... die Translate von $Z$ im wesentlichen disjunkt sind, d.h. $Z \cap \tau(Z)$ ist eine Seitenfläche, eine Kante, ein Eckpunkt des Polyeders oder komplett leer, wenn $\tau\in T\setminus\set{id}$ ist.
\end{enumerate}
Die Menge $M$ nennt man \udot{Motiv} des Kristallgitters.

Eine Basiszelle, in der $Z$ das kleinstmöglichen Volumen hat, heißt \udot{elementare Basiszelle} des Gitters.
\end{definition}

\begin{remark}[Von Basiszellen zurück zu Kristallen]
Meistens betrachtet man einen Parallelepiped als Basiszelle, d.h. man nimmt drei unabhängige Vektoren $v_1,v_2,v_3$ und betrachtet
\[Z:=\Set{\lambda_1v_1+\lambda_2v_2+\lambda_3v_3 | 0\leq\lambda_1,\lambda_2,\lambda_3\leq 1}\]
zusammen mit einem irgendeinem Motiv $M\subseteq Z$. Daraus lässt sich dann wieder ein Kristall bauen, indem man die Translationen entlang der Vektoren $v_1,v_2,v_3$ wiederholt anwendet.

Achtung: Der so erhaltende Kristall hat $(Z,M)$ als Basiszelle, aber i.A. nicht als elementare Basiszelle, es könnte eine kleinere Zelle geben. $v_1,v_2,v_3$ müssen auch keine Basis des Translationsgitters sein. Es könnte z.B. sein, dass es $\frac{1}{42}v_1+\frac{1}{7}v_2$ auch eine Translation des so erhaltenen Gitters ist.
\end{remark}
