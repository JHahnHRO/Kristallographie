% !TeX spellcheck = de_DE
% !TeX root = kristallographie_skript.tex
Es gibt genau eine Bewegungsgruppe in Null Dimensionen, nämlich die triviale Gruppe $\set{1}$, denn das ist gleich der vollen Bewegungsgruppe $Isom(\IR^0)$.

\begin{theorem}[Dimension 1]
Es sei $G\leq\Isom(\IR^1)$ eine endliche Gruppe von eindimensionalen Bewegungen. Dann ist $G$ eine der folgenden Gruppen:
\begin{enumerate}
\item Die triviale Gruppe $\set{\id}$.
\item Spiegelungen: Es gibt einen Punkt $x\in\IR$, sodass $G$ genau aus der Identität und der Spiegelung an $x$ besteht.
\end{enumerate}
\end{theorem}

\begin{theorem}[Dimension 2]
Es sei $G\leq\Isom(\IR^2)$ eine endliche Gruppe von zweidimensionalen Bewegungen. Dann ist $G$ eine der folgenden Gruppen:
\begin{enumerate}
\item Drehgruppen: Es gibt einen Punkt $x\in\IR^2$ und eine natürliche Zahl $n\in\IN_{\geq 1}$, sodass $G$ genau aus den Drehungen mit Drehmittelpunkt $x$ um die Winkel $0,\frac{2\pi}{n}, \frac{2}{n}2\pi, \ldots, \frac{(n-1)}{n}2\pi$ besteht.
\item Diedergruppen: Es gibt einen Punkt $x\in\IR^2$ ein $n\in\IN_{\geq 1}$, sodass $G$ genau aus den $n$ Drehungen mit Drehmittelpunkt $x$ um die Winkel $0,\frac{2\pi}{n}, \frac{2}{n}2\pi, \ldots, \frac{(n-1)}{n}2\pi$ und $n$ Spiegelungen an Geraden, die sich allesamt in $x$ schneiden und mit $\frac{2\pi}{n}$-Winkel Abstand angeordnet sind.
\end{enumerate}
\end{theorem}

\begin{theorem}[Klassifikation der endlichen Drehgruppen in drei Dimensionen]
Sei $G\leq\Isom(\IR^3)$ eine endliche Gruppe von orientierungserhaltenden dreidimensionalen Bewegungen. Dann ist $G$ eine der folgenden Gruppen:

\begin{enumerate}
\item Zyklische Gruppen, von einer Drehung erzeugt: Es gibt Gerade $A\subseteq\IR^3$ und eine natürliche Zahl $n\in\IN_{\geq 1}$, sodass $G$ genau aus den $n$ Drehungen mit Drehachse $A$ um die Winkel $0,\frac{2\pi}{n}, \frac{2}{n}2\pi, \ldots, \frac{(n-1)}{n}2\pi$.

\item Diedergruppen, von Rotationen erzeugt: Es eine Gerade $A\subseteq\IR^3$, eine dazu senkrechte Ebene $E\leq\IR^3$ und eine natürliche Zahl $n\in\IN_{\geq 2}$, sodass $G$ genau aus den $n$ Drehungen mit Drehachse $A$ um die Winkel $0,\frac{2\pi}{n}, \frac{2}{n}2\pi, \ldots, \frac{(n-1)}{n}2\pi$ sowie aus $n$ 180°-Drehungen mit Drehachsen in $E$, die sich alle im Punkt $E\cap A$ schneiden und regelmäßig im $\frac{2\pi}{n}$ Winkel-Abstand angeordnet sind, besteht.

\item Die Drehgruppe eines regelmäßigen Tetraeders.

\item Die Drehgruppe eines Würfels / Oktaeders.

\item Die Drehgruppe eines Ikosaeders / Dodekaeders.
\end{enumerate}

\end{theorem}
\begin{proof}
Schritt 0: Nullpunkt festlegen.

Jede endliche Bewegungsgruppe fixiert mindestens einen Punkt. Wählen wir nämlich ein beliebiges $x\in\IR^3$, dann ist
\[\frac{1}{\abs{G}}\sum_{g\in G} g(x)\]
ein Punkt, der ein gemeinsamer Fixpunkt aller Elemente von $G$ ist.

Einen solchen globalen Fixpunkt erklären wir zum Nullpunkt eines Koordinatensystems. Wir wissen jetzt aus der Klassifikation, dass jede Bewegung, die den Nullpunkt festlässt, eine Drehung oder Ebenenspiegelung oder Drehinversion ist. Spiegelungen und Drehinversionen sind nicht Orientierungserhaltend, also besteht $G$ aus Drehungen um Achsen, die sich im Nullpunkt schneiden.

\medbreak
Schritt 1: Eine Gruppenoperation finden.

$G$ erhält Längen, d.h. es operiert auf $S^2:=\Set{x\in\IR^3 | \norm{x}=1}$. Wir betrachten darin die Menge
\[\Omega:=\Set{x\in S^2 | \exists g\in G\setminus\set{1}: g(x)=x}\]
der Fixpunkte von nichttrivialen Elementen.

$G$ operiert wirklich auf $\Omega$, denn wenn $g(x)=x$ gilt und $h\in G$ ist, dann ist $h(x)$ ein Fixpunkt von $hgh^{-1}$.

\medbreak
Schritt 2: Zählen.

Da jede Drehung außer der um 0° die Punkte auf der Drehachse fixiert, hat jedes Element $g\in G$ genau zwei Fixpunkte der Länge 1. Die Identität hat natürlich genau $\abs{\Omega}$ Fixpunkte.

Es sei $r:=\abs{\Omega/G}$ die Anzahl der Bahnen der Operation von $G$ auf $\Omega$. Und sei $p_1,\ldots,p_r$ je ein Punkt aus jeder Bahn und $a_i:=\abs{G_{p_i}}$ die Größe der dazugehörigen Stabilisatoren. Aufgrund des Satzes von Burnside gilt also:
\begin{align*}
r &= \frac{1}{\abs{G}}\sum_{g\in G} \abs{Fix(g)} \\
&=\frac{\abs{\Omega} + (\abs{G}-1)2}{\abs{G}} \\
&=\frac{\sum_{i=1}^r \abs{^G p_i}}{\abs{G}} + 2(1-\frac{1}{\abs{G}}) \\
&=\frac{1}{a_1}+\frac{1}{a_2}+\cdots+\frac{1}{a_r}+2(1-\frac{1}{\abs{G}})
\end{align*}
Umgestellt:
\[2-\frac{2}{\abs{G}} = \sum_{i=1}^r(1-\frac{1}{a_i}) \iff \frac{2}{\abs{G}}+(-2+r) =  \sum_{i=1}^r \frac{1}{a_i} \]

\medbreak
Schritt 3: Fallunterscheidungen.

Hierbei sind $\abs{G}$ und $a_i$ natürliche Zahlen. Außerdem ist $a_i>1$, da nach Definition $p_i$ ja von mindestens einem Element außer der Identität fixiert wird, und $a_i$ außerdem ein Teiler von $\abs{G}$.

Die Summanden $1-\frac{1}{a_i}$ sind also alle $\geq 1-\tfrac{1}{2}$ und $2-\frac{2}{\abs{G}}$ ist eine Zahl $<2$. Also kann es maximal drei Summanden geben.

\smallbreak
Fall 3.1.: $r=0$.

Das tritt nur ein, wenn $\Omega=\emptyset$ ist, d.h. wenn es überhaupt keine nichttrivialen Elemente gibt, d.h. wenn $G=\set{1}$ ist. Das ist eine Gruppe vom Typ a.

\smallbreak
Fall 3.2.: $r=1$.

Dann muss $\abs{G}=a_1b_1$ sein für ein $b_1\in\IN$ und die Gleichung $2-\frac{2}{\abs{G}} = 1-\frac{1}{a_1}$ ist äquivalent zu $a_1+1 = \frac{2}{b_1}$. Links steht eine ganze Zahl größer gleich $3$, rechts eine Zahl kleiner 2. Das ist unmöglich.

\smallbreak
Fall 3.3.: $r=2$.

Die rechte Gleichung ist dann $\frac{2}{\abs{G}} = \frac{1}{a_1} + \frac{1}{a_2}$ und  nur erfüllbar, wenn $a_1$ und $a_2$ gleich $\abs{G}$ sind, denn $a_1,a_2$ sind ja Teiler von, also kleiner gleich $\abs{G}$.

Alle Elemente der Gruppe fixieren also $p_1$ und $p_2$. Da mit $p_1$ auch $-p_1$ ein Fixpunkt ist, muss $p_2=-p_1$ sein. Also fixiert die Gruppe auch die Gerade durch $p_1$ und $-p_1$, also den eindimensionalen Unterraum $U=\IR p_1$. Da $G$ aus orthogonalen Transformationen besteht, werden rechte Winkel erhalten, d.h. $G$ bildet die Ebene $U^\perp$, die aus allen Vektoren besteht, die senkrecht zu $U$ sind, in sich ab. Es handelt sich somit um eine zweidimensionale, orientierungserhaltende Bewegungsgruppe. Das ist eine Drehgruppe.

\smallbreak
Fall 3.4.: $r=3$.

O.B.d.A. nehmen wir an, dass $a_1\geq a_2\geq a_3$ ist. Die rechte Gleichung ist dann $\frac{2}{\abs{G}}+1=\frac{1}{a_1}+\frac{1}{a_2}+\frac{1}{a_3}$.

Fall 3.4.1.: $a_1=a_2=a_3=2$. Dann ist $\abs{G}=4$. Da orthogonale Abbildungen linear sind, ist ${^G(-p)} = -(^G p)$, d.h. wenn $B$ eine Bahn ist, ist $-B$ auch eine Bahn. Da es nur drei Bahnen gibt, muss mindestens eine davon $B=-B$ erfüllen, d.h. diese Bahn besteht aus zwei antipodalen Punkten, die von einem Element von $G$ vertauscht werden. OBdA ist das die erste Bahn. Die Gerade $A$ durch $\pm p_1$ ist somit $G$-invariant.

Fall 3.4.2.: $a_1>2$ und $a_2=a_3=2$. Dann ist $\abs{G}=2n$, wobei $n=a_1$. Dann gibt es genau eine Bahn mit $2$ Punkten, nämlich ${^Gp_1}$, die dann also ein Paar von Antipoden sein müssen. Die Gerade $A$ durch diese beiden Punkte wird also von $G$ auf sich selbst abgebildet und von einigen Elementen in der Richtung geflippt.

In 3.4.1 und 3.4.2 finden wir also eine invariante Gerade $A$. Die Elemente des Stabilisators $G_{p_1}$ sind $n$ Drehungen um $A$. Die Elemente, die nicht im Stabilisator liegen, müssen $p_1$ und $-p_1$ vertauschen, sind also 180°-Drehungen um Achsen, die senkrecht zu $A$ sind. Das sind die anderen $n$ Elemente. Wir haben also eine in beiden Fällen eine Diedergruppe gefunden.

Fall 3.4.3: $(a_1,a_2,a_3)=(3,3,2)$. Dann ist $\abs{G}=12$. Die vier Punkte in der ersten bzw. zweiten Bahn bilden jeweils einen regelmäßigen Tetraeder, in dessen Symmetriegruppe $G$ einbettet.

Fall 3.4.4: $(a_1,a_2,a_3)=(4,3,2)$. Dann ist $\abs{G}=24$. Die sechs Punkte in der ersten Bahn bilden die Eckpunkte eines Oktaeders und die acht Punkte in der zweiten Bahn bilden die Eckpunkte eines Würfels, in dessen Symmetriegruppen $G$ jeweils einbettet.

Fall 3.4.5: $(a_1,a_2,a_3)=(5,3,2)$. Dann ist $\abs{G}=60$. Die 12 Punkte der ersten Bahn bilden die Eckpunkte eines Dodekaeders, die 20 Punkte der zweiten Bahn bilden die Eckpunkte eines Ikosaeders, in dessen Symmetriegruppen $G$ jeweils einbettet.
\end{proof}

\begin{remark}
Jede dieser Gruppen kommt als Drehgruppe eines Polyeders vor. Die Diedergruppen sind Drehgruppen von $n$-eckigen, geraden Prismen. Die zyklischen Gruppen sind z.B. die Drehgruppen von $n$-eckigen Pyramiden.
\end{remark}

\begin{remark}
\emph{Nicht} jede dieser Gruppen kommt auch in den Symmetrien eines dreidimensionalen Kristalls vor, z.B. gibt es keine Kristalle mit Ikosaeder/Dodekaeder-Symmetrie und keine, die Drehungen der Ordnung 5 oder $\geq 7$ enthalten. Nur Drehungen der Ordnung 2,3,4 und 6 treten in Kristallen auf. Wir werden noch sehen, wieso.
\end{remark}