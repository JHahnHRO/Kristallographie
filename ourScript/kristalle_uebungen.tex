% !TeX spellcheck = de_DE
% !TeX root = kristallographie_skript.tex
\begin{sheet}

\begin{problem}
Bestimme die möglichst viele wesentlich verschiedene Symmetrien der folgenden 2D Kristalle ...
\end{problem}

\begin{problem}[difficulty={fortgeschritten}]
Das Prinzip, mit dem wir die Operation von $\Aut(\Lambda)$ auf einem Drei-Torus konstruiert haben, lässt sich wie folgt abstrahieren und verallgemeinern:

Gegeben eine Gruppe $G$, einen Normalteiler $N\unlhd G$ und eine Operation von $G$ auf $\Omega$, dann operiert $G/N$ auf dem Bahnenraum $\Omega/N$ wie folgt:
\[{^{gN} B} := {^g B} := \Set{{^g \omega} | \omega\in B}\]
\begin{subproblem}[difficulty={einfach}]
Man zeige, dass dies wohldefiniert ist, d.h.
\[\forall g,h\in G, B\in\Omega/N: gN=hN \implies {^g B} = {^h B}\]
\end{subproblem}
\begin{subproblem}[difficulty={mittel}]
Man überlege sich, dass dies genau unsere Konstruktion mit dem 3-Torus ergibt, wenn wir $G=\Aut(\Lambda)$ auf $\Omega=\IR^3$ operieren lassen und $N=\mathcal{T}$ betrachten. Hinweis: Man überlege sich zuerst, wieso der 3-Torus genau $\IR / \mathcal{T}$ ist.
\end{subproblem}
\end{problem}

\end{sheet}