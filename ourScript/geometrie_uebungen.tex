\begin{sheet}

\begin{problem}[difficulty={Für $n=2$ relativ leicht, für $n=3$ mittelschwer, für $n>3$ mittelschwer bis schwer ohne Algebra}]
Wir haben affine Abbildung algebraisch durch die Gleichung
\[f(\lambda p_1 + (1-\lambda)p_0) = \lambda f(p_1)+(1-\lambda)f(p_0)\]
definiert, es gibt aber auch eine äquivalente, rein geometrische Definition. Beweise:

Eine bijektive Abbildung $f:\IR^n\to\IR^n$ für $n\geq 2$ ist affin genau dann, wenn $f$ Geraden auf Geraden abbildet.

Hinweis: Translationen haben definitiv diese Eigenschaft, d.h. man kann o.B.d.A. nur Abbildungen mit $f(0)=0$ betrachten und zeigen, dass sie linear sind.
\end{problem}

\begin{problem}[title={Komposition von Spiegelungen},difficulty={leicht bis mittel}]
Beweise, dass die Hintereinanderausführung von zwei Ebenenspiegelungen im $\IR^3$...
\begin{subproblem}
... an zwei parallelen Ebenen eine Translation ist. Um welchen Verschiebungsvektor?
\end{subproblem}
\begin{subproblem}
... an zwei sich schneidenden Ebenen eine Drehung ist. Um welche Achse und welchen Drehwinkel?
\end{subproblem}
\end{problem}

\begin{problem}[title={Komposition von Drehungen},difficulty={Schwerer als es scheint}]
Beweise, dass die Hintereinanderausführung von zwei Drehungen im $\IR^3$, deren Drehachsen sich in einem Punkt schneiden, wieder eine Drehung ist, deren Achse die anderen beiden im selben Punkt schneidet.

Was passiert, wenn die Drehachsen sich nicht schneiden?
\end{problem}

\begin{problem}[title={Konjugation geometrisch}, difficulty={mittel}]
Es seien $D,S,I,T$ je eine Drehung, eine Ebenenspiegelung, eine Drehinversion und eine Translation. Es sei $F$ eine beliebige Bewegung. Zeige:
\begin{subproblem}
$F\circ D\circ F^{-1}$ ist wieder eine Drehung. Um welche Achse und welchen Winkel?
\end{subproblem}
\begin{subproblem}
$F\circ S\circ F^{-1}$ ist wieder eine Ebenenspiegelung. An welcher Ebene?
\end{subproblem}
\begin{subproblem}
$F\circ I\circ F^{-1}$ ist wieder eine Drehinversion. An welcher Achse und mit welchem Drehwinkel?
\end{subproblem}
\begin{subproblem}
$F\circ T\circ F^{-1}$ ist wieder eine Translation. Um welchen Verschiebungsvektor?
\end{subproblem}
\end{problem}


\end{sheet}