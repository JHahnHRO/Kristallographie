Jetzt, wo wir wissen, welche sieben verschiedenen Gittersysteme es gibt, können wir uns für jedes Gittersystem eine Basiszelle aussuchen, welche wir zur Standardbasiszelle deklarieren. Ferner können wir die nun auch mit Hilfe der Klassifikation eine Symbolik einführen, welche eindeutig festlegt, welche Symmetrieoperationen und mit welcher Orientierung zueinander in dem Kristallsystem vorhanden sind - die \udot{Hermann-Mauguin-Symbolik}. Ein Hermann-Mauguin-Symbol besteht aus bis zu 3 Symbolen. Die Symbole geben an, welche Symmetrieoperation in dem Kristallsystem vorhanden ist, die Stelle, an der das Symbol steht, gibt Aufschluss über die Orientierung der Symmetrieoperation im Verhältnis zur Basiszelle. Dazu werden für jede der 7 Basiszellen bis zu 3 "Blickrichtungen" (inklusive ihrer Reihenfolge) definiert. Symmetrieoperationen, deren richtungsgebende Vektoren (bei Drehungen/Schraubungen die Dreh-/Schraubachse, bei Spiegelungen/Gleitspiegelungen der Normalenvektor) parallel zu einer  Blickrichtung sind, werden in einem Symbol zusammengefasst und an die entsprechende Stelle der Blickrichtung vermerkt. Die Symbole können sowohl Symmetrieoperationen von Punktgruppen als auch von Raumgruppen beschreiben. Dabei gilt folgendes:
\begin{itemize}
	\item Tritt eine Drehung/Schraubung auf, so schreibe die Zahl der Drehung/Schraubung - z.B. $4$, $4_1$, aber auch $\overline{4}$
	\item Tritt eine Spiegelung auf, so schreibe $m$, bei Gleitspiegelungen z.B. $a$
	\item Tritt eine Drehung und eine Spiegelung auf (die Drehung ist dann senkrecht zur Spiegelung), so schreibe einen Bruch, z.B. $\frac{4}{m}$
	\item Tritt in einer Blickrichtung keine Symmetrie auf, so kann entweder $1$ geschrieben werden, oder gar nichts
\end{itemize}
Auf der folgenden Seite sind 6 der 7 Basiszellen zusammen mit ihren Blickrichtungen dargestellt (das trikline Gitter hat keine Blickrichtung, da ein trikliner Kristall höchstens die Inversionssymmetrie aufweist).

\includepdf[pages={81-86},nup=2x3, frame=true]{../Grundkrist_2013_inFarbe_mini.pdf}

