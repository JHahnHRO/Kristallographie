
\begin{definition}[Kristalle]
Ein \udot{Kristall} (auch \udot{Kristallgitter}) ist eine Punktmenge $\Lambda\subseteq\IR^3$ (gedacht als die Menge aller Atome im Kristall), die ...
\begin{enumerate}
\item ... Translationssymmetrie hat, d.h. es gibt Vektoren $t_1,t_2,t_3\in\IR^3$ in drei unabhängige Richtungen, sodass immer, wenn $x\in\Lambda$ ein Punkt im Kristall ist, $x+k_1 t_1+k_2t_2+k_3t_3$ auch ein Punkt im Kristall ist für alle ganzen Zahlen $k_1,k_2,k_3\in\IZ$.
\item ... aus isolierten Punkten besteht, d.h. es gibt einen Mindestabstand $\delta>0$, sodass sich keine zwei Punkte $x,y\in\Lambda$ näher als $\delta$ kommen: $x\neq y \implies\norm{x-y}\geq\delta$.
\end{enumerate}
\end{definition}

\begin{remark}
Insbesondere bedeutet die Bedingung des Mindestabstands, dass es nur abzählbar viele Punkte im Gitter gibt.
\end{remark}

\begin{definition}
Die \udot{Symmetriegruppe} eines Kristalls $\Lambda$ ist die Gruppe aller starren (=abstandserhaltenden) Bewegungen, die das Gitter in sich selbst abbilden:
\[\Aut(\Lambda) := \Set{s\in \operatorname{Isom}(\IR^3) | s(\Lambda)=\Lambda}\]
Nach Definition enthält $\Aut(\Lambda)$ mindestens die drei Translationen $x\mapsto x+t_i$. Die Menge \emph{aller} Translationen, die $\Lambda$ in sich selbst abbilden, sind eine Untergruppe von $\Aut(\Lambda)$.
\end{definition}

\begin{definition}
Ist $\Lambda$ ein Kristall, dann ist
\[Trans(\Lambda) := \Set{v\in\IR^3 | \forall a\in\Lambda: a+v\in\Lambda}\]
das \udot{Translationsgitter} des Kristalls.
\end{definition}

\begin{remark}
Weil die Punkte in $\Lambda$ einen Mindestabstand haben, ist die Translationsuntergruppe diskret, d.h. sie enthält eine Basis: Drei Translationen $\tau_1,\tau_2,\tau_3\in\Aut(\Lambda)$, sodass sich jede beliebige Translation $\tau\in\Aut(\Lambda)$ auf eindeutige Weise als $\tau_1^{k_1}\circ\tau_2^{k_2} \circ \tau_3^{k_3}$ mit $k_1,k_2,k_3\in\IZ$ schreiben lässt.

Die Translationsuntergruppe ist also zur Gruppe $(\IZ^3,+)$ isomorph.
\end{remark}

\begin{example}
Umgekehrt: Wenn $v_1,v_2,v_3\in\IR^3$ drei beliebige, linear unabhängige Vektoren sind, dann ist $\Lambda:=\IZ v_1+\IZ v_2 + \IZ v_3 = \Set{k_1 v_1+k_2 v_2+k_3 v_3 | k_1,k_2,k_3\in\IZ}$ ein Gitter, dessen Translationsuntergruppe die drei Translationen $\tau_i:=x\mapsto x+v_i$ als (eine mögliche von vielen) Basis hat.
\end{example}

\begin{definition}
Es sei $\Lambda$ ein Kristallgitter und $T$ die Gruppe aller Translationen, die $\Lambda$ invariant lassen.

Eine \udot{Basiszelle von $\Lambda$} ist ein Paar $(Z,A)$ bestehend aus einem (konvexer, kompakter) Polyeder $Z\subseteq\IR^3$ und einer Punktmenge $M\subseteq Z$, sodass
\begin{enumerate}
\item ... die Translate von $M$ ganz $\Lambda$ überdecken, d.h. $\Lambda=\bigcup_{t\in T} t(M)$.
\item ... die Translate von $Z$ ganz $\IR^3$ überdecken, d.h. $\IR^3 = \bigcup_{t\in T} t(Z)$.
\item ... die Translate von $Z$ im wesentlichen disjunkt sind, d.h. $Z \cap t(Z)$ ist leer oder höchstens eine Seitenfläche, Kante oder Eckpunkt des Polyeders, wenn $t\in T\setminus\set{id}$ ist.
\end{enumerate}
Die Menge $M$ nennt man \udot{Motiv} des Kristallgitters.

Eine Basiszelle, in der $Z$ das kleinstmöglichen Volumen hat, heißt \udot{elementare Basiszelle} des Gitters.
\end{definition}

\begin{remark}
Da $Z$ kompakt ist und die Punkte in $\Lambda$ einen Mindestabstand haben, muss $M=Z\cap\Lambda$ endlich sein.
\end{remark}
